\documentclass[journal]{IEEEtran}
\usepackage[T1]{fontenc}
\usepackage[utf8]{inputenc}
\usepackage{lmodern}
\usepackage[polish]{babel}
\usepackage{makeidx}
\usepackage{amsfonts}
\usepackage{graphicx}
\usepackage{url}
\usepackage{hyperref}

\title{Dodatkowe aplikacje w aparatach słuchowych: wystrzałowe gadżety czy przydatne ułatwienia}
\author{Bartłomiej~Bułat, Tomasz~Drzewiecki}

\begin{document}

%\markboth{Sztuczne Sieci Neuronowe, semestr letni 2011/2012, prowadzący: mgr
%inż. Tomasz Orzechowski}{}

\maketitle


\begin{abstract}
W~tym artykule są opisane historyczne oraz współczesne aparaty słuchowe. Główny nacisk jest położony na coraz częście pojawiających różnych dodatkowych funkcjach w~tych urządzeniach.
Przedstawiono krótką historię aparatów słuchowych, od skromnych początków do najnowszych generacji, które pozwalają się podłączyć technologią bezprzewodową do innych urządzeń elektrycznych.
\end{abstract}

\begin{IEEEkeywords}
Aparaty słuchowe, innowacje, gadżet
\end{IEEEkeywords}

\section{Wstęp}

Ubytek słuchu to dla każdego człowieka utrudnienie życiowe. Słuchu używamy często w~codziennym życiu. Jest on nieodzowny podczas komunikacji z~innymi ludźmi a~także w poznawaniu otaczającego nas świata. Również człowiek stworzył muzykę, której jedynym zadaniem jest sprawienie człowiekowi przyjemności, której przy znacznym ubytku słuchu można być pozbiawonym.

Z~wyżej wymienionych powodów ludzie starali się poprawiać słuch, dlatego już od XVII wieku tworzone pierwsze urządzenia mające na celu pomoc osobom z~wadą słuchu.

Dopiero rozwój elektroniki pozwoli na konstruowanie bardziej zaawansowanych rozwiąniań, dlatego pierwsze elektryczne aparaty słuchwe powstały dopiero w XX wieku.

Przez lata doskonalono urządzenia i obecnie technologia pozwala na bardzo wiele. Obecnie poza doskonaleniem działania układów mających na celu przekazywać dźwięki z~otoczenia do ucha rozwija się technologie wspomagającie działania aparatów, na których skupia się ten artykuł.

\section{Historia aparatów słuchowych}

Pierwszymi urządzeniami skonstruowanymi w~celu poprawy jakości słyszenia były trąbki, których węższy koniec przykładano do ucha.

Pierwsze prace nad zastosowaniem sygnałów eletrycznych w~celu poprawy słyszanie przez ludzi z~ubytkiem słuchu realizował Graham Bell. Nie udało mu się osiągnąć celu, lecz na podstawie przeprowadzonych badań stworzył telefon.

W XX wieku skonstruowano pierwsze wzmacniacze elektryczne, których można było użyć do konstrukcji elektrycznych urządzeń - już aparatów słuchowych. Pierwsze maszyny były dużych rozmiarów zarówno ze względu na samo urządzenie, jak i~z~uwagi na baterię. Postępująca miniaturyzacja pozwoliła na stworzenie urządzeń kieszonkowych w~latach czterdziestych XX wieku. Kolejne generacje aparatów słuchowych coraz bardzie upodabniały się do tych znanych nam obecnie. W~latach sześćdziesiątych stworzono pierwszy aparat zauszny, obecnie najczęściej spotykany.

Późniejsze aparaty stawały się coraz mniejsze oraz coraz lepiej działające. Tworzono lepsze procesory dźwięku, z~lepszymi algorytmami przetwarzania sygnału oraz z~coraz popularniejszymi dodatkowymi aplikacjami.

\section{Współczesne aparaty słuchowe}

Obecnie apraty słuchowe służą nie tylko umożliwieniu odbioru dźwięków lub poprawie słuchu. Choć jest to ich główna funkcja producenci urządzeń starają się zachęcić pacjentów dodatkowymi aplikacjami, które poprawiają działanie urządzenia (np. usuwanie sprzężeń zwrotnych), mają na celu poprawę komfortu użytkowania (np. tłumienie szumu) lub (np. interpolacja dźwięku przestrzennego)

Poniżej są przedstawione najczęściej spotykane dodatki.

\subsection{Tłumienie szumu}

Prawie wszystkie modele renomowanych firm dostępne na rynku oferują tłumienie szumu i~redukcję hałasu. Ta funkcja jest dość pożądana w aparacie słuchowym z~uwagi na możliwy długi i~trudny proces dopoasowywania aparatu do pacjenta oraz uczenia się pacjenta użycia urządzenia. Dotyczy to zwłaszcza pacjentów, którzy wcześniej mieli znaczny ubytek lub całkowity brak słuchu. Odbiór dźwięku z~użyciem aparatu może odbiegać od tego, do czego pacjent był przyzwyczajony wcześniej. Dlatego korzystne jest ograniczenie ilości i głośności dochodzących dźwięku, więc taka funkcja jest bardzo przydatna.

Dodatkową funkcją jest redukcja szumu spowodowanego przez wiatr. Nie jest jednak ona oferowana w~tak szerokiej gamie modeli, jak podstawowa redukcja szumu. Nie jest to do końca uzasadnione, ponieważ szum wiatru wiejącego w~mikrofon może bardzo utrudnić słyszenie. Być może jest to tylko element marketingowy i~podstawowa funkcja tłumienia szumu pozwoli na rozumienie mowy przy wietrze. Niemniej funkcja tłumienia szumu pochodząca od wiatru jest przydatna i~powinna być obecna w~zdecydowanej większości aparatów.

\subsection{Usuwanie sprzężeń zwrotnych}

Również ta funkcja jest obecna w~większości modeli. Jest także istotna z punktu widzenia pacjenta. Podczas użytkowania aparatu słuchowego mogą powstawać sprzężenia zwrotne, które są bardzo nieprzyjemne. Dla niektórych pacjentów, zwłaszcza przy długim procesie dopasowania do apratu może być to odstraszające od używania urządzenia, mimo wszystkich jego zalet. Dlatego należy ocenić tą funkcję jako pożądaną. 

\subsection{Rożne tryby działania}

W mniejszej ilości aparatów można spotkać funkcję modyfikacji działania urządzenia w~zależności od otoczenia. Pozwala to na dokładne słyszenie w~różnych warunkach. Jest to próba odwzorowania sposobu działania naturalnego ludzkiego słuchu, które pozwala człowieku na dokładne rozumienie mowy w różnych sytuacjach. W~związku z~tym tą funkcję także można uznać za przydatną. Fakt, że jest ona obecna w mniejszej ilości aparatów pokazuje, że bez tego również mozna sprawnie funkcjonować.

%\subsection{R

\section{Podsumowanie}

Podsumowanie

\section{Dodatek A: }

Ewentualne dodatki


\begin{thebibliography}{123}
    \bibitem{google}Google - uniwersalne źródło wiedzy,
        \url{http://www.google.com}
\end{thebibliography}

\end{document}
