\documentclass[journal]{IEEEtran}
\usepackage[T1]{fontenc}
\usepackage[utf8]{inputenc}
\usepackage{lmodern}
\usepackage[polish]{babel}
\usepackage{makeidx}
\usepackage{amsfonts}
\usepackage{graphicx}
\usepackage{url}
\usepackage{hyperref}

\title{Dodatkowe aplikacje w aparatach słuchowych: wystrzałowe gadżety czy przydatne ułatwienia}
\author{Bartłomiej~Bułat, Tomasz~Drzewiecki}

\begin{document}

%\markboth{Sztuczne Sieci Neuronowe, semestr letni 2011/2012, prowadzący: mgr
%inż. Tomasz Orzechowski}{}

\maketitle


\begin{abstract}
Abstrakt
\end{abstract}

\begin{IEEEkeywords}
Keywords
\end{IEEEkeywords}

\section{Wstęp}

Wstęp

\section{Aparaty słuchowe}

Treść

\section{Podsumowanie}

Podsumowanie

\section{Dodatek A: }

Ewentualne dodatki


\begin{thebibliography}{123}
    \bibitem{google}Google - uniwersalne źródło wiedzy,
        \url{http://www.google.com}
s\end{thebibliography}

\end{document}
